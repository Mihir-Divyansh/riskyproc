%%%%%%%%%%%%%%%%%%%%%%%%%%%%%%%%%%%%%%%%%
% University Assignment Title Page 
% LaTeX Template
% Version 1.0 (27/12/12)
%
% This template has been downloaded from:
% http://www.LaTeXTemplates.com
%
% Original author:
% WikiBooks (http://en.wikibooks.org/wiki/LaTeX/Title_Creation)
%
% License:
% CC BY-NC-SA 3.0 (http://creativecommons.org/licenses/by-nc-sa/3.0/)
% 
% Instructions for using this template:
% This title page is capable of being compiled as is. This is not useful for 
% including it in another document. To do this, you have two options: 
%
% 1) Copy/paste everything between \begin{document} and \end{document} 
% starting at \begin{titlepage} and paste this into another LaTeX file where you 
% want your title page.
% OR
% 2) Remove everything outside the \begin{titlepage} and \end{titlepage} and 
% move this file to the same directory as the LaTeX file you wish to add it to. 
% Then add \input{./title_page_1.tex} to your LaTeX file where you want your
% title page.
%
%%%%%%%%%%%%%%%%%%%%%%%%%%%%%%%%%%%%%%%%%
%\title{Title page with logo}
%----------------------------------------------------------------------------------------
%	PACKAGES AND OTHER DOCUMENT CONFIGURATIONS
%----------------------------------------------------------------------------------------

\documentclass[12pt]{article}
\usepackage[english]{babel}
\usepackage[utf8x]{inputenc}
\usepackage{amsmath}
\usepackage{graphicx}
\usepackage[colorinlistoftodos]{todonotes}
\usepackage{listings}
\usepackage{float}

\begin{document}

\begin{titlepage}

\newcommand{\HRule}{\rule{\linewidth}{0.5mm}} % Defines a new command for the horizontal lines, change thickness here

\center % Center everything on the page
 
%----------------------------------------------------------------------------------------
%	HEADING SECTIONS
%----------------------------------------------------------------------------------------

\textsc{\LARGE Indian Institute of Technology }\\[0.3cm] 
\textsc{\Large Hyderabad }\\[0.3cm]

%----------------------------------------------------------------------------------------
%	TITLE SECTION
%----------------------------------------------------------------------------------------

\HRule \\[0.4cm]
{ \huge \bfseries CS2323 HW2}\\[0.03cm] % Title of your document
\HRule \\[0.7cm]

 
%----------------------------------------------------------------------------------------
%	AUTHOR SECTION
%----------------------------------------------------------------------------------------

\textbf{Mihir Divyansh E, EE23BTECH11017} \\[0.5cm]


% If you don't want a supervisor, uncomment the two lines below and remove the section above
%\Large \emph{Author:}\\
%John \textsc{Smith}\\[3cm] % Your name

%----------------------------------------------------------------------------------------
%	DATE SECTION
%----------------------------------------------------------------------------------------

{\large August-November, 2025}\\[1cm] % Date, change the \today to a set date if you want to be precise

%----------------------------------------------------------------------------------------
%	LOGO SECTION
%----------------------------------------------------------------------------------------

% \includegraphics{logo.png}\\[1cm] % Include a department/university logo - this will require the graphicx package
 
%----------------------------------------------------------------------------------------

\vfill % Fill the rest of the page with whitespace

\end{titlepage}

\section{Question 1} 

\textbf{Write equivalent machine code (in hexadecimal) for the given assembly instructions, by highlighting the various fields in the 32-bits of the instruction:}
\begin{enumerate}
    \item addi x15, x22, -45\\
    This instruction uses I-Format encoding. The imm value here is -45, which in 2's complement form is 111111010011. 
    \\
        \begin{tabular}{|c|c|c|c|c|}
            \hline
            Immediate & rs1 & Funct3 & rd & Opcode \\
            \hline
            111111010011 & 10110 & 000 & 01111 & 0010011\\
            \hline
            &&0xFD3B0793&& \\
            \hline
        \end{tabular}
    \item and x23, x8, x9
    \\
        This instruction uses R-Format encoding. \\
    \\
        \begin{tabular}{|c|c|c|c|c|c|}
            \hline
            Funct7 & rs2 & rs1 & Funct3 & rd & Opcode \\
            \hline
            0000000 & 01001 & 01000 & 111 & 10111 & 0110011\\
            \hline
            &&&0x00947BB3&& \\
            \hline
        \end{tabular}
    \item blt x2, x11, 240
    \\ 
    This instruction uses B-Format encoding.
    The imm value here is 240, which in binary is 000011110000. Rearranged, it becomes 0000001111000000. We take the 12th bit to be imm[12], the next 6 bits (000011) to be imm[10:5] and the last 5 bits (11000) to be imm[4:1,11] and the last bit is imm[11].\\
        \begin{tabular}{|c|c|c|c|c|c|}
            \hline
            Immediate[12,10:5] & rs2 & rs1 & Funct3 & Immediate[4:1,11] & Opcode \\
            \hline
            00001110 & 01011 & 00010 & 100 & 10000 & 1100011\\
            \hline
            &&&0x0EB14863&& \\
            \hline
        \end{tabular}
    \item sd x19, -54(x1)
        \\
        This is a S-Format instruction. -54 in 2's complement format : 111111001010. 
        \begin{tabular}{|c|c|c|c|c|c|}
            \hline
            Immediate[11:5] & rs2 & rs1 & Funct3 & Immediate[4:0] & Opcode \\
            \hline
            1111110 & 10011 & 00001 & 011 & 01010 & 0100011\\
            \hline
            &&&0xFD30B523&& \\
            \hline
        \end{tabular}
    \item jal x3, -10116
        \\
        This is a J-Format instruction. -10116 in 2's complement format for 21 bits : 111111101100001111100. \\
        \begin{tabular}{|c|c|c|c|}
            \hline
            Immediate[20,10:1] & Immediate[11,19:12] & rd & Opcode \\
            \hline
            10000111110 & 111111101 & 00011 & 1101111\\
            \hline
            &0x87DFD1EF&& \\
            \hline
        \end{tabular}
\end{enumerate}

\section{Question 2}
\textbf{For various pseudo instructions shown below, write their equivalent using a maximum of 2 real instructions.}

\begin{enumerate}
    \item \todo[inline, color = red!10]{li x5, 0xFFFFFFFFFFFFFFFF}
    \begin{lstlisting}
    addi x5, x0, -1    # Sign extension automatic
    \end{lstlisting}
    \item \todo[inline, color = red!10]{li x5, 132}
    \begin{lstlisting}
    addi x5, x0, 132
    \end{lstlisting}
    \item \todo[inline, color = red!10]{li x5, 2134}
    \begin{lstlisting}
    addi x5, x5, 2047
    addi x5, x5, 87
    \end{lstlisting}
    \item \todo[inline, color = red!10]{li x5, 0x000000002345abcd}
    \begin{lstlisting}
    lui x5, 0x2345a     # Load upper 20 bits
    addi x5, x5, 0xbcd  # load lower 12 bits
    \end{lstlisting}
\end{enumerate}

\section{Question 3}
\textbf{Convert the given instructions in hex to their corresponding assembly code}

\begin{enumerate}
    \item 0x0019F233 : This in binary is 0000000\_00001\_10011\_111\_00100\_0110011
        \begin{itemize}
            \item 0110011 is opcode 
            \item So, the instruction is R format.
            \item Funct3 and Funct7 are 111 and 0000000 respectively, which means the operation is or.
            \item rs2 is 1, rs1 is 19, rd is 4.
            \item Therefore the instruction would be \todo[inline, color = orange!20]{and, x4, x19, x1}
        \end{itemize}
    \item 0x06B4D763 : This in binary is 0000011\_01011\_01001\_101\_01110\_1100011
        \begin{itemize}
            \item 1100011 is opcode
            \item So, the instruction is B format.
            \item The immediate value is 0000001101110, which is 110 in decimal.
            \item Funct3 is 101, which means the operation is bge.
            \item rs2 is 11, rs1 is 9.
            \item Therefore the instruction would be \todo[inline, color = orange!20]{bge, x9, x11, 110}
        \end{itemize}
    \item 0x0169CF93 : This in binary is 000000010110\_10011\_100\_11111\_0010011
        \begin{itemize}
            \item 0010011 is opcode
            \item So, the instruction is I format.
            \item Funct3 is 100, which means the operation is xori.
            \item rs1 is 19, rd is 31.
            \item The immediate value is 0000000010110, which is 22 in decimal.
            \item Therefore the instruction would be \todo[inline, color = orange!20]{xori, x31, x19, 22}
        \end{itemize}
\end{enumerate}

\end{document}