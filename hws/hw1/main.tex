%%%%%%%%%%%%%%%%%%%%%%%%%%%%%%%%%%%%%%%%%
% University Assignment Title Page 
% LaTeX Template
% Version 1.0 (27/12/12)
%
% This template has been downloaded from:
% http://www.LaTeXTemplates.com
%
% Original author:
% WikiBooks (http://en.wikibooks.org/wiki/LaTeX/Title_Creation)
%
% License:
% CC BY-NC-SA 3.0 (http://creativecommons.org/licenses/by-nc-sa/3.0/)
% 
% Instructions for using this template:
% This title page is capable of being compiled as is. This is not useful for 
% including it in another document. To do this, you have two options: 
%
% 1) Copy/paste everything between \begin{document} and \end{document} 
% starting at \begin{titlepage} and paste this into another LaTeX file where you 
% want your title page.
% OR
% 2) Remove everything outside the \begin{titlepage} and \end{titlepage} and 
% move this file to the same directory as the LaTeX file you wish to add it to. 
% Then add \input{./title_page_1.tex} to your LaTeX file where you want your
% title page.
%
%%%%%%%%%%%%%%%%%%%%%%%%%%%%%%%%%%%%%%%%%
%\title{Title page with logo}
%----------------------------------------------------------------------------------------
%	PACKAGES AND OTHER DOCUMENT CONFIGURATIONS
%----------------------------------------------------------------------------------------

\documentclass[12pt]{article}
\usepackage[english]{babel}
\usepackage[utf8x]{inputenc}
\usepackage{amsmath}
\usepackage{graphicx}
\usepackage[colorinlistoftodos]{todonotes}
\usepackage{listings}

\begin{document}

\begin{titlepage}

\newcommand{\HRule}{\rule{\linewidth}{0.5mm}} % Defines a new command for the horizontal lines, change thickness here

\center % Center everything on the page
 
%----------------------------------------------------------------------------------------
%	HEADING SECTIONS
%----------------------------------------------------------------------------------------

\textsc{\LARGE Indian Institute of Technology }\\[0.3cm] 
\textsc{\Large Hyderabad }\\[0.3cm]

%----------------------------------------------------------------------------------------
%	TITLE SECTION
%----------------------------------------------------------------------------------------

\HRule \\[0.4cm]
{ \huge \bfseries CS2323 HW1}\\[0.03cm] % Title of your document
\HRule \\[0.7cm]

 
%----------------------------------------------------------------------------------------
%	AUTHOR SECTION
%----------------------------------------------------------------------------------------

\textbf{Mihir Divyansh E, EE23BTECH11017} \\[0.5cm]


% If you don't want a supervisor, uncomment the two lines below and remove the section above
%\Large \emph{Author:}\\
%John \textsc{Smith}\\[3cm] % Your name

%----------------------------------------------------------------------------------------
%	DATE SECTION
%----------------------------------------------------------------------------------------

{\large August-November, 2025}\\[1cm] % Date, change the \today to a set date if you want to be precise

%----------------------------------------------------------------------------------------
%	LOGO SECTION
%----------------------------------------------------------------------------------------

% \includegraphics{logo.png}\\[1cm] % Include a department/university logo - this will require the graphicx package
 
%----------------------------------------------------------------------------------------

\vfill % Fill the rest of the page with whitespace

\end{titlepage}

\section{Question 1} 

\textbf{Write an assembly instruction to achieve the given functionality, defined using C-language syntax}
\begin{enumerate}
    \item x8 = x5 - 5 \\ This can be achieved using the instruction: \todo[inline, color=blue!20]{sub x8, x5, 5} It subtracts 5 from the value in register x5 and stores the result in register x8.
    \item x5 = x3 * 8 \\ Use \todo[inline, color=blue!20]{mul x5, x3, 8} x3 is multiplied by 8, and the result is stored in x5.
    \item x19 += x10 \\ Use \todo[inline, color=blue!20]{add x19, x19, x10} This adds the value in register x10 to the value in register x19 and stores the result in register x19.
    \item ++x15 \\ This is an increment operation is done using: \todo[inline, color=blue!20]{add x15, x15, 1} It adds 1 to the value in register x15 and stores the result in register x15.
    \item x9 = x15/4 \\ This is a shift operation. \todo[inline, color=blue!20]{srl x9, x15, 2} It performs a logical right shift on the value in register x15 by 2 bits, effectively dividing it by 4, and stores the result in register x9.
    \item x12 = x19 + 24 \\ We can use add with immediate instruction: \todo[inline, color=blue!20]{addi x12, x19, 24} 
\end{enumerate}

\section{Question 2}
\textbf{Consider an array M consisting of 8 byte integers. The base address of M is stored in register x5. Write the assembly code that achieves each operation given below.}

\begin{enumerate}
    \item M[12] = M[20] + 100
\begin{lstlisting}
addi x6, x5, 160   # Load address of M[20] into x6
ld x7, 0(x6)       # Load M[20] into x7
addi x7, x7, 100   # Add 100 to x7
addi x6, x5, 96    # Load address of M[12] into x6
sd x7, 0(x6)       # Store result into M[12]
\end{lstlisting}
    \item M[20] ++
\begin{lstlisting}
ld x7, 160(x5)     # Load M[20] into x7
addi x7, x7, 1     # Increment x7
sd x7, 160(x5)     # Store result back into M[20]
\end{lstlisting}
    \item swap M[5] and M[12]
\begin{lstlisting}
ld x6, 40(x5)
add x7, x0, x6     # copy from x6 to x7
ld x6, 96(x5)
sd x7, 96(x5)
sd x6, 40(x5)
\end{lstlisting}
    \item Make the first 32-bits (from MSB side) of M[4] as 0
\begin{lstlisting}
ld x6, 32(x5)
slli x6, x6, 32     # Shift left by 32 bits
slri x6, x6, 32     # Shifting back will -
sd x6, 32(x5)       # make the first 32 bits 0
\end{lstlisting}
    \item Swap the most significant 32-bits of M[2] with its least significant 32-bits
\begin{lstlisting}
ld x6, 16(x5)       
slli x7, x6, 32     
srli x6, x6, 32
or x6, x6, x7
sd x6, 16(x5)
\end{lstlisting}

\end{enumerate}

\section{Question 3}
\textbf{Write the following decimal numbers in their 2's complement representation, using 8-bits. Show your calculations.}
\begin{enumerate}
    \item +23
    \\ The unsigned binary representation of 23 is $\{\lfloor\frac{23}{2^7}\rfloor, \lfloor\frac{23}{2^6}\rfloor, \lfloor\frac{23}{2^5}\rfloor, \lfloor\frac{23}{2^4}\rfloor, \lfloor\frac{23}{2^3}\rfloor, \lfloor\frac{23}{2^2}\rfloor, \lfloor\frac{23}{2^1}\rfloor, \lfloor\frac{23}{2^0}\rfloor\} = \{0, 0, 0, 1, 0, 1, 1, 1\}$
    which is the same as it's 2's complement representation as it is positive.
    \item -1 
    \\ The binary representation of 1 is $\{0, 0, 0, 0, 0, 0, 0, 1\}$. To find the 2's complement, we invert the bits and add 1 to it to get $\{1, 1, 1, 1, 1, 1, 1, 0\} + 1 = \{1, 1, 1, 1, 1, 1, 1, 1\}$. 
    \item +255
    \\ 255 requires 8 unsigned bits to represent. Since we have 8 bits signed, we cannot represent 255 in 8-bit signed.
    \item -128
    \\ in 2's complement, -128 is represented as $\{1, 0, 0, 0, 0, 0, 0, 0\}$. 
\end{enumerate}

\section{Question 4}
\textbf{ Write the equivalent decimal number for given numbers in 2's complement format. Show your calculations}
\begin{enumerate}
    \item 11010100
    \\ $-1\cdot2^7 + 1\cdot2^6 + 0\cdot2^5 + 1\cdot2^4 + 0\cdot2^3 + 1\cdot2^2 + 0\cdot2^1 + 0\cdot2^0 = -44$
    \item 00101011
    \\ $0\cdot2^7 + 0\cdot2^6 + 1\cdot2^5 + 0\cdot2^4 + 1\cdot2^3 + 0\cdot2^2 + 1\cdot2^1 + 1\cdot2^0 = 43$
    \item 11111110
    \\ $-1\cdot2^7 + 1\cdot2^6 + 1\cdot2^5 + 1\cdot2^4 + 1\cdot2^3 + 1\cdot2^2 + 1\cdot2^1 + 0\cdot2^0 = -2$
\end{enumerate}
\end{document}